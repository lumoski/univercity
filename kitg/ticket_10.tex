\section{Отношение нестрогого и строгого, линейного порядка: определение, важнейшие примеры. 
Определить, является ли отношение отношением порядка.}

Пусть $A$ -- множество, $R$ -- отношение на множестве $A$, т.е. $R \subseteq A^2$.

\begin{definition}
    Отношение $R$ называется отношением \textit{нестрогого частичного
    порядка} на множестве $A$, если $R$: рефлексивное, антисимметричное,
    транзитивное. (Отношение $a \leq b$)
\end{definition}

\begin{definition}
    Если $R$ -- порядок на множестве $A$ и $a,b \in R$ или $aRb$, то
    элементы $a$ и $b$ называются \textit{сравнимыми}.
\end{definition}

Порядок называют частичным порядком, так как не обязательно все элементы
являются сравнимыми.

\begin{definition}
    Отношение $R$ называется отношением \textit{строгого частичного
    порядка} на множестве $A$, если оно является антирефлексивным,
    асимметричным и транзитивным.
\end{definition}

В \textit{строгом частичном порядке} можно ограничиться двумя свойствами:
антирефлексивность, транзитивность. (доказывается через предположение
отсутствия асимметрии, а это противоречит антирефлексивности).

\begin{definition}
    Порядок $R$ называется линейным порядком, если выполняются
    условия: $$(\forall a,b \in A)(a=b \vee aRb \vee bRa).$$
\end{definition}

Другими словами, все различные элементы в линейном порядке обязательно
сравнимы. Поэтому в случае, когда порядок линейный, слово частичный
опускается.

На множестве $Z$ отношение “меньше или равно” является нестрогим
линейным порядком.