\section{Отношение эквивалентности: определение, примеры. Классы эквивалентности, полная система 
представителей. \\ Фактор-множество множества по отношению эквивалентности. Алгоритм 
выделения классов эквивалентности по графу отношения. Проверить, является ли указанное 
отношение отношением эквивалентности.}

\begin{definition}
    Бинарное отношение $R$, заданное на множестве $A$, называется
    \textit{отношением эквивалентности}, если оно рефлексивно, симметрично и
    транзитивно.
\end{definition}

Пример:
\begin{align*}
    A - \text{множество граждан России} \\
    A = \set{x | x - \text{Гражданин РФ}} \\
    R = \set{(x, y) | x,y \in A; x, y - \text{родились в одном месяце}} \\
    R - \text{рефлексивно, симметрично, транзитивно} \Rightarrow \\
    \Rightarrow R - \text{отношение эквивалентности}
\end{align*}

Пусть $R$ -- отношение эквивалентности на множестве $A$ и элемент $a \in A$.

\begin{definition}
    Классом эквивалентности отношения эквивалентности $R$,
    порождённым элементом $a$ (обозначается $\overline{a}$), называется множество всех
    элементов множества $A$, которые находятся в отношении $R$ с элементом $a$.
    \begin{align*}
        \overline{a} = \set{x \in A | xRa}
    \end{align*}
\end{definition}

\begin{definition}
    Любой элемент класса эквивалентности называется
    \textit{представителем этого класса}.
\end{definition}

\begin{definition}
    \textit{Полной системой представителей классов эквивалентности}
    называется множество представителей всех классов, взятых по одному и
    только по одному из каждого класса эквивалентности.
\end{definition}

\begin{definition}
    Пусть $A$ -- непустое множество. Фактор-множеством
    множества $A$ по отношению эквивалентности $R$ называется множество всех
    классов эквивалентности.
    \begin{align*}
        A/R=\set{\overline{\overline{a}} | a \in A}.
    \end{align*}
\end{definition}

Алгоритм выделения классов эквивалентности по графу отношения:
\begin{enumerate}[left=0.0em, labelsep=1em, topsep=0.5em, itemsep=0pt, parsep=0.5em]
    \item Всем вершинам графа приписываем значение 0: color(v):=0
    \item k:=0 (количество классов эквивалентности)
    \item Рассматриваем вершины графа
\end{enumerate}

Если есть вершина, у которой color(v)=0, то k:=k+1, этой вершине и всем,
смежным с ней, присваиваем значение k: color(v):=k, выводим эти вершины с
их цветом. \\
Если вершин, у которых color(v)=0, нет, то выводим значение k – число
классов эквивалентности и stop. \\
Результатом работы алгоритма является число k – количество классов
эквивалентности; каждой вершите графа присвоен номер класса
эквивалентности. \\

Временная сложность алгоритма зависит от представления графа. Если
применена матрица смежности, то временная сложность равна $O(n^2)$, а если
нематричное представление –- $O(n+n)$ = $O(2^n)$: рассматриваются все вершины и часть ребер.