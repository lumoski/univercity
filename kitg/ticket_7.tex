\section{Свойства бинарных отношений: рефлексивность, антирефлексивность, симметричность, 
антисимметричность, асимметричность, транзитивность. Свойства матриц и графов таких 
отношений, число таких отношений, заданных на n-элементном множестве.}

Будем рассматривать бинарные отношения на множестве $A^2$ в этом случае говорят что бинарное отношение задано на множестве $A$.

\begin{definition}
	Бинарное отношение $R$, заданное на множестве $A$ называется \textit{рефлексивным}, если для любого $a \in A$ верно $aRa$.
	\begin{align*}
		R - \textit{рефлексивно} \iff (\forall a \in A) \; aRa
	\end{align*}
	Если $R$ рефлексивное отношение, то квадратная матрица размера $n$ на главной диагонали состоит только из 1.
    В графе рефлексивное бинарное отношение в каждой вершине имеет \textit{петлю}.
\end{definition}

\begin{definition}
	Бинарное отношение $R$, заданное на множестве $A$ называется \textit{антирефлексивным}, если для любого $a \in A$ неверно $aRa$.
	\begin{align*}
		R - \textit{антирефлексивно} \iff (\forall a \in A) \; \overline{aRa}
	\end{align*}
    Если $R$ антирефлексивное отношение, то квадратная матрица размера $n$ на главной диагонали состоит только из 0.
    В графе антирефлексивное бинарное отношение в каждой вершине не имеет \textit{петель} вообще.
\end{definition}

\begin{definition}
	Бинарное отношение $R$, заданное на множестве $A$ называется \textit{симметричным} если для любого $a,b \in A$ верно $aRb \Rightarrow bRa$.
	\begin{align*}
		R - \textit{симметрично} \iff (\forall a, b \in A) \; aRb &\Rightarrow bRa\\
		a \; \textit{может быть} &= b
	\end{align*}
    Матрица симметричного бинарного отношения симметрична относительна главной диагонали.
    Граф симметричного бинарного отношения \textit{как правило} неориентированный.
\end{definition}

\begin{definition}
	Бинарное отношение $R$, заданное на множестве $A$ называется \textit{асиметричным}, если для любого $a, b \in A$ верно $aRb$, то неверно $bRa$ ($b\overline{R}a$).
	\begin{align*}
		R - \textit{асимметрично} \iff (\forall a,b \in A) \; aRb \Rightarrow \overline{bRa}
	\end{align*}
    В матрице бинарного отношения элементы симметричные главной диагонали \textit{различны}.
    В графе если есть ребро $ab$, то нет ребра $ba$.
\end{definition}

\begin{definition}
	Бинарное отношение $R$, заданное на множестве $A$ называется \textit{антисимметричным}, если для любого $a,b \in A$, $aRb \wedge bRa \Rightarrow a = b$
	\begin{align*}
		R - \textit{антисимметрично} \iff (\forall a, b \in A) \; (aRb \wedge bRa) \Rightarrow a = b
	\end{align*}
	Элементы матрицы антисимметричного бинарного отношения относительно главной диагонали \textit{различны}. На главной диагонали могут быть 1.
	В графе при антисимметричном бинарном отношении могут быть \textit{петли}
\end{definition}

\begin{definition}
	Бинарное отношение $R$, заданное на множестве $A$ называется \textit{транзитивным}, если для любого $a, b, c \in A$, ($aRb \wedge bRc$) $\Rightarrow aRc$.
	\begin{align*}
		R - \textit{транзитивно} \iff (\forall a, b, c \in A) \; (aRb \wedge bRc) \Rightarrow aRc
	\end{align*}
\end{definition}

Если отношение обладает каким-нибудь свойством, нужно доказать это, если
не обладает, то нужно привести контрпример.

% TODO:
Упражнение 1. Найти число рефлексивных отношений n-элементного
множества \\
Упражнение 2. Найти число антирефлексивных отношений n-элементного
множества \\
Упражнение 3. Найти число симметричных отношений n-элементного
множества \\
Упражнение 4. Найти число антисимметричных отношений n-элементного
множества \\
Упражнение 5. Найти число асимметричных отношений n-элементного
множества \\
% ----