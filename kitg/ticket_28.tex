\section{Цикломатическое число графа, теорема о связи цикломатического числа с количеством вершин, 
ребер и компонент связности.}

\begin{definition}
    \textit{Цикломатическим числом графа} называется минимальное
    число рёбер, которое нужно удалить, чтобы в графе не было циклов.
\end{definition}

Если граф $G$ -- связный, то нужно удалить рёбра так, чтобы получилось
остовное дерево графа.

\begin{theorem}
    Цикломатическое число графа, у которого $n$ вершин, $r$ рёбер и $k$
    компонент связности, равно $r - n + k$.
\end{theorem}

\begin{proof}
    Рассмотрим сначала случай, когда граф $G$ -- связный, у него
    $n$ вершин и $r$ рёбер, одна компонента связности. Для того, чтобы у подграфа
    этого графа не было циклов, нужно удалить рёбра так, чтобы получилось
    остовное дерево графа.

    Пусть $s$ -- количество рёбер, которые нужно удалить, чтобы получить остовное
    дерево. Из общего числа ребер $r$ вычитаем число ребер в остовном дереве.
    Известно, что их $n - 1$.
    \begin{align*}
        s = r - (n - 1) = r - n + 1 \text{-- цикломатическое число связного графа}.
    \end{align*}
    Обозначим цикломатическое число графа $\mu(G)$.

    Пусть теперь $G(V, E)$ -- граф, у которого $n$ вершин, $r$ рёбер и $k$ компонент
    связности. Тогда нам нужно получить остовной лес. Для этого в каждой
    компоненте связности ищем его остовное дерево. Обозначим $n_i$ и $r_i$ --
    количество вершин и ребер в компоненте связности с номером $i$. Тогда
    цикломатическое число равно:
    \begin{multline*}
        \mu(G) = (r_1 - n_1 +1) + (r_2 - n_2 + 1) + \dots + (r_k - n_k + 1) =\\
        = (r_1 + r_2 + \dots + r_k) - (n_1 + n_2 + \dots + n_k) + k = r - n + k.
    \end{multline*}
\end{proof}