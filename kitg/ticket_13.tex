\section{Связность в неориентированном графе. Теорема о разложении графа в объединение связных 
подграфов. Компоненты связности графа. Алгоритм нахождения связных компонент графа. 
Найти компоненты связности указанного графа.}

\begin{definition}
    Две вершины в графе называются \textit{связанными}, если существует
    маршрут с началом в первой вершине и концом во второй.
\end{definition}

\begin{definition}
    Граф называется \textit{связным}, если любые две его вершины
    являются связанными.
\end{definition}

\begin{theorem}
    В неориентированном графе отношение связанности на множестве
    вершин является отношением эквивалентности.
    \begin{align*}
        V, \rho = \set{(v_1,v_2),v_1,v_2 \in V, \text{$v_1$ и $v_2$ -- \textit{связанные}}}
    \end{align*}
    \begin{enumerate}[left=0.0em, labelsep=1em, topsep=0.5em, itemsep=0pt, parsep=0.5em]
        \item Рефлексивность -- $\exists$ нуль-маршрут ($\forall v \in V, \; v \rho v$)
        \item Симметричность -- $\exists$ маршрут $ue_1 \dots e_sv$ ($u \rho v$) +
        $\exists$ маршрут $ve_s \dots e_1u$ ($v \rho u$)
        \item Транзитивность -- $\forall u,v,w \in V \; u \rho v \land v \rho w \Rightarrow u \rho w$.
        $\exists$ маршрут $u \dots v$ и $v \dots w \Rightarrow \exists$ маршрут $u \dots w$
    \end{enumerate}
\end{theorem}

\begin{definition}
    Пусть $G(V, E)$ -- граф, то граф $G(V_1,E_1)$ -- подграф графа G, если
    $V_1 \subset V \text{ и } E_1 \subset V$.
\end{definition}

\begin{theorem}
    Любой неориентированный граф распадается в объединение своих
    связных подграфов.
\end{theorem}

\begin{proof}
    Доказано, что в неориентированном графе отношение
    связанности на множестве вершин является отношением эквивалентности.\\
    Известно, что отношение эквивалентности разбивает множество на
    непересекающиеся подмножества - классы эквивалентности, поэтому всё
    множество вершин разбивается на попарно непересекающиеся подмножества
    $V_1,V_2,\dots,V_s, V=V_1 \cup V_2 \cup \dots \cup V_s$.\\
    В каждом таком подмножестве $V_i$ вершины являются связанными. Вершины
    из разных подмножеств не являются связанными. Это, в частности, означает,
    что в графе нет ребер, инцидентных вершинам из разных компонент
    связности. Это и доказывает справедливость утверждения.
\end{proof}

Связные подграфы также называются \textit{связными компонентами графа}.

\textbf{Задача 1.} Доказать, что любой конечный граф имеет чётное количество число
вершин нечётной степени.\\
\textbf{Задача 2.} Доказать, что если в графе только две вершины имеют нечётную
степень, то они связанные.
\begin{proof}
    Докажем, что две вершины, которые имеют нечетную степень, принадлежат
    одной компоненте связности. Предположим противное. Пусть эти вершины
    принадлежат разным компонентам связности. Тогда в подграфах, содержащих
    эти вершины, содержится только одна вершина нечётной степени, что
    противоречит задаче 1.\\
    Поэтому эти вершины принадлежат одной компоненте связности, тогда они
    являются связанными.
\end{proof}

Для нахождения количество связных компонент графа существует алгоритм:
Применяем алгоритм обхода графа в ширину.
\begin{enumerate}[left=0.0em, labelsep=1em, topsep=0.5em, itemsep=0pt, parsep=0.5em]
    \item Всем вершинам графа приписываем значение 0: color(v):=0
    \item k:=0 (количество компонент связности)
    \item Рассматриваем вершины графа.
    Если есть вершина, у которых color(v)=0, то k:=k+1, присваиваем этой
    вершине color(v):=k, заносим вершину в очередь.
    \item Посещается первая вершина из очереди. Всем смежным с ней
    вершинам, у которых color(v)=0, присваиваем color(v):=k и заносим
    в очередь. После этого первая вершина в очереди удаляется из очереди.
    \item Переходим к шагу 4 до тех пор, пока очередь не пуста.
    \item Переходим к шагу 3 до тех пор, пока есть вершины, у которых
    color(v)=0.
    \item Если вершин, у которых color(v)=0, нет, то выводим значение k – число
    компонент связности и stop.
\end{enumerate}

Результатом работы алгоритма является число k -- количество компонент
связности; каждой вершине графа присвоен номер компоненты связности

Иначе говоря: количество запусков алгоритмов в ширину для обхода всего графа
= количеству компонент связности.