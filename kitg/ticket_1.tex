\section{Основные определения теории графов. Смежность и инцидентность вершин и ребер графа. 
Степени вершин в графе и орграфе. Теоремы о сумме степеней вершин в графе и орграфе. 
Матрицы смежности и инцидентности. Найти матрицы смежности и инцидентности указанного 
графа.}

$V$ -- множество вершин. \\
$E$ -- множество пар вида $(u, v): u, v \in V$ (множество ребер).

\begin{definition}
    Графом -- называют совокупность 2-ух множеств непустого множества E
	\begin{equation*}
		E = \{(u, v): u, v \in V\}, \; G(V, E) - \textit{обозначение графа}
	\end{equation*}
\end{definition}

\begin{definition}
    Петля -- пара вида $(v,v)$ в множестве $E$.
\end{definition}

\begin{definition}
    Кратные ребра -- одинаковые пары в множестве $E$. Количество кратных ребер - кратность ребра.
\end{definition}

\noindent
Существуют следующие виды графов:
\begin{enumerate}[left=0.0em, labelsep=1em, topsep=0.0em, itemsep=0pt, parsep=0.5em]
    \item Псевдограф -- в графе могут быть и \textit{кратные ребра}, и \textit{петли}.
    \item Мультиграф -- в графе есть \textit{кратные ребра}, но нет \textit{петель}.
    \item Простой граф -- отсутствуют и \textit{кратные ребра} и \textit{петли}.
\end{enumerate}

\begin{definition}
    Ориентированный граф (орграф) -- граф с ориентированными ребрами.
\end{definition}

\begin{definition}
    Если $e=(u,v)$ -- ребро неориентированного графа, то $u, v$ - концы ребра.
\end{definition}

\begin{definition}
    Если $e=(u,v)$ -- ребро (дуга) ориентированного графа, то $u$ - начало ребра, $v$ - конец ребра.
\end{definition}

\begin{definition}
    $a, b \text{ \textit{смежные}} \Leftrightarrow e=(a,b)$.
\end{definition}

\begin{definition}
    $u, v \text{ \textit{инцидентны ребру }} e \Leftrightarrow e=(u,v)$.
\end{definition}

\begin{definition}
	Степенью вершины $v$ неориентированного графа называется количество ребер инцидентных данной вершине, $\delta(v)$ (петлю считают два \mbox{раза}).
\end{definition}

\begin{minipage}{0.55\textwidth}
	\centering
	\begin{tikzpicture}
		\node[main] (1) {$1$};
		\node[main] (3) [above right of=1] {$3$};
		\node[main] (5) [below right of=3] {$5$};
		\node[main] (2) [above of=5] {$2$};
		\node[main] (4) [below left of=5] {$4$};
		
		\draw (3) to [out=90,in=180,looseness=5] (3);
		\draw (1) -- (3);
		\draw (3) -- (5);
		\draw (1) -- (5);
		\draw (4) -- (5);
	\end{tikzpicture}
\end{minipage}
\begin{minipage}{0.3\textwidth}
	$\begin{aligned}
		&\delta(1) = 2 & \delta(4) &= 1 \\
		&\delta(2) = 0 & \delta(5) &= 3 \\
		&\delta(3) = 4 & \Rightarrow sum &= 10
	\end{aligned}$
\end{minipage}

\begin{theorem}
    Сумма степеней вершин неориентированного графа равна удвоенному числу ребер
	\begin{align*}
		\sum_{u \in V}\delta(v)=2r, \textit{ где r - число ребер}
	\end{align*}
\end{theorem}

\begin{proof}
    Теорема справедлива, так как вклад каждого ребра равен двум.
\end{proof}

\begin{definition}
    Если степень вершины равна нулю, то вершина \textit{изолированная}, \\ $\delta(v)=0$.
\end{definition}

\begin{definition}
    Если степень вершины равна единице, то вершина \textit{висячая}, $\delta(v)=1$
\end{definition}

\begin{definition}
    Полустепенью исхода(захода) вершина $v$ ориентированного графа называют количество ребер исходящих(заходящих) в данную вершину.
    \begin{tikzpicture}[baseline={(A.base)}]
		\node [text width=5cm] (A) at (0,0) {$\delta^-(v)$ -- полустепень исхода\\
			$\delta^+(v)$ -- полустепень захода};
	\end{tikzpicture}
\end{definition}

\begin{theorem}
	Для орграфа справедливо равенство
	\begin{align*}
		\sum_{u \in V}\delta^-(v)=\sum_{u \in V}\delta^+(v)=r, \textit{ где r -- число ребер}
	\end{align*}
\end{theorem}

\begin{minipage}{0.5\textwidth}
	\centering
	\begin{tikzpicture}
		\node[main] (a) {$a$};
		\node[main] (d) [right of=a] {$d$};
		\node[main] (b) [above of=d] {$b$};
		\node[main] (c) [right of=b] {$c$};
		\node[main] (e) [below of=d] {$e$};
		
		\draw[-{Stealth[length=2mm]}] (e) to [out=270,in=360,looseness=5] (e);
		\draw[-{Stealth[length=2mm]}] (a) -- (d);
		\draw[-{Stealth[length=2mm]}] (a) -- (e);
		\draw[-{Stealth[length=2mm]}] (a) -- (b);
		\draw[-{Stealth[length=2mm]}] (b) -- (c);
		\draw[-{Stealth[length=2mm]}] (e) -- (d);
	\end{tikzpicture}
\end{minipage}
\begin{minipage}{0.5\textwidth}
	$\begin{aligned}
		\delta^-(a) &= 3 &\delta^+(a) = 0 \\
		\delta^-(b) &= 1 &\delta^+(b) = 1 \\
		\delta^-(c) &= 0 &\delta^+(c) = 1 \\
		\delta^-(d) &= 0 &\delta^+(d) = 2 \\
		\delta^-(e) &= 2 &\delta^+(e) = 2 \\
		\sum\delta^-(v) &= 6; \; &\sum\delta^+(v) = 6 \\
	\end{aligned}$
\end{minipage}

\begin{definition}
    Матрицей смежности графа(орграфа) называют квадратную матрицу размерностью $n$, где $n=|v|$ (мощность множества вершин), в котором $a_{ij}=k$, где $k$ - число ребер $(v_i,v_j)$
\end{definition}

\begin{definition}
    Пусть $G(V,E)$ -- неориентированный граф. Матрицей инцидентности неориентированного графа называется матрица $B$ размером $n*r$, $|v| = n, |E| = r$, где каждый элемент матрицы:
	\begin{align*}
		b_{ij}=\left\{\begin{array}{l}
			1, \text{ если $v_i$ инцидентно ребру $e_j$}
			\\ 
			0, \text{ иначе}
		\end{array}\right.
	\end{align*}
\end{definition}
\noindent
Такая матрица будет симметричной.

\begin{definition}
    Пусть $G(V,E)$ -- ориентированный граф. Матрицей инцидентности ориентированного графа называется матрица $B$ размером $n*r$, $|v| = n, |E| = r$, где каждый элемент матрицы:
	\begin{align*}
		b_{ij}=\left\{\begin{array}{l}
			-1, \text{ если ребро $e_j$ выходит из $v_i$}
			\\
			1, \text{ если ребро $e_j$ входит из $v_i$}
			\\ 
			0, \text{ иначе}
		\end{array}\right.
	\end{align*}
\end{definition}

Если есть петля, то на соответствующее место ставят любое число.

Поиск матрицы смежности $A$ и инцидентности $B$ для неориентированного графа:
\begin{figure}[ht]
	\centering
	\begin{minipage}[b]{0.45\linewidth}
		\centering
		\begin{tikzpicture}[node distance=30mm]
			\node[main] (a) {$a$};
			\node[main] (b) [above of=a] {$b$};
			\node[main] (c) [right of=b] {$c$};
			\node[main] (d) [below of=c] {$d$};
			
			\draw[] (a) -- node[midway, left]{$e_1$} (b);
			\draw[] (a) -- node[midway, below right]{$e_2$} (c);
			\draw[] (b) to [out=30, in= 150] node[midway, above]{$e_3$} (c);
			\draw[] (b) -- node[midway, below]{$e_4$} (c);
			\draw[] (c) -- node[midway, right]{$e_5$} (d);
			\draw[] (d) to [out=270,in=360,looseness=5] node[midway, below right]{$e_6$} (d);
		\end{tikzpicture}
	\end{minipage}
	\begin{minipage}[b]{0.45\linewidth}
		\begin{flalign*}
			&A =
			\begin{blockarray}{ccccc}
				& a & b & c & d \\
				\begin{block}{c(cccc)}
					a & 0 & 1 & 1 & 0 \\
					b & 1 & 0 & 2 & 0 \\
					c & 1 & 2 & 0 & 1 \\
					d & 0 & 0 & 1 & 1 \\
				\end{block}
			\end{blockarray} \\
			&B =
			\begin{blockarray}{ccccccc}
				& e_1 & e_2 & e_3 & e_4 & e_5 & e_6 \\
				\begin{block}{c(cccccc)}
					a & 1 & 1 & 0 & 0 & 0 & 0\\
					b & 1 & 0 & 1 & 1 & 0 & 0\\
					c & 0 & 1 & 1 & 1 & 1 & 0\\
					d & 0 & 0 & 0 & 0 & 1 & 1\\
				\end{block}
			\end{blockarray}
		\end{flalign*}
	\end{minipage}
\end{figure}

\newpage
Поиск матрицы смежности $A$ и инцидентности $B$ для ориентированного графа:
\begin{figure}[ht]
	\centering
	\begin{minipage}[b]{0.40\linewidth}
		\begin{tikzpicture}[node distance=22mm, minimum size=17, remember picture, overlay]
			\node[main] (a) at ([xshift=-5.2cm, yshift=5cm]current page) {$a$};
			\node[main] (b) [above right of=a] {$b$};
			\node[main] (c) [right of=b] {$c$};
			\node[main] (d) [right of=a] {$d$};
			\node[main] (e) [below right of=a] {$e$};
			
			\draw[-{Stealth[length=2mm]}] (e) to [out=270,in=360,looseness=5] node[midway, below]{$e_6$}(e);
			\draw[-{Stealth[length=2mm]}] (a) -- node[midway, above]{$e_2$} (d);
			\draw[-{Stealth[length=2mm]}] (a) -- node[midway, below left]{$e_3$} (e);
			\draw[-{Stealth[length=2mm]}] (a) -- node[midway, left]{$e_1$} (b);
			\draw[-{Stealth[length=2mm]}] (b) -- node[midway, below]{$e_4$} (c);
			\draw[-{Stealth[length=2mm]}] (e) -- node[midway, left]{$e_5$} (d);
		\end{tikzpicture}
		\hfil
	\end{minipage}
	\begin{minipage}[t]{0.45\linewidth}
		\vspace*{-5mm}
		\begin{flalign*}
			&A =
			\begin{blockarray}{cccccc}
				& a & b & c & d & e\\
				\begin{block}{c(ccccc)}
					a & 0 & 1 & 0 & 1 & 1 \\
					b & 0 & 0 & 1 & 0 & 0 \\
					c & 0 & 0 & 0 & 0 & 0 \\
					d & 0 & 0 & 0 & 0 & 0 \\
					e & 0 & 0 & 0 & 1 & 1 \\
				\end{block}
			\end{blockarray} \\
			&B =
			\begin{blockarray}{ccccccc}
				& e_1 & e_2 & e_3 & e_4 & e_5 & e_6 \\
				\begin{block}{c(cccccc)}
					a & -1 & -1 & -1 & 0 & 0 & 0\\
					b & 1 & 0 & 0 & -1 & 0 & 0\\
					c & 0 & 0 & 0 & 1 & 0 & 0\\
					d & 0 & 1 & 0 & 0 & 1 & 0\\
					e & 0 & 0 & 1 & 0 & -1 & 1\\
				\end{block}
			\end{blockarray}
		\end{flalign*}
	\end{minipage}
\end{figure}