\section{Деревья. Теорема о связи вершин и ребер в дереве.}

\begin{definition}
    Граф называется \textit{деревом}, если он является связным и не
    содержит циклов.
\end{definition}

\begin{definition}
    Граф, все компоненты связности которого являются деревьями,
    называется \textit{лесом}.
\end{definition}

\begin{theorem}
    Пусть граф $G$ -- дерево, у которого $n$ вершин и $r$ ребер. Тогда
    справедливо равенство $r = n - 1$.
\end{theorem}

\begin{proof}
    Методом математической индукции:
    \begin{enumerate}[left=0.0em, labelsep=1em, topsep=0.0em, itemsep=0pt, parsep=0.5em]
        \item $G$ -- дерево. $n=1$. Так как петель в дереве не может быть, то $r=0$.
        \item Пусть равенство выполняется для дерева с $n = k$ вершинами. То есть у
        такого дерева $r = k - 1$ ребер. Докажем, что равенство будет выполняться
        и для дерева $G$ с $k+1$ вершиной.
        \item В дереве $G$ существует ребро, поэтому есть и висячая вершина. Эта
        вершина не является шарниром, поэтому после ее удаления вместе с
        инцидентным ему ребром граф останется связным. Очевидно также, что
        в полученном после удаления вершины и ребра графе не будет циклов
        (так как их нет в $G$). Поэтому полученный граф является деревом с k
        вершинами. Известно, по предположению индукции, что у него $k - 1$
        ребро. У графа $G$ на одну вершину и на одно ребро больше, поэтому y
        него $k+1$ вершина и $k$ ребер, то есть число ребер на единицу меньше
        числа вершин.
        \item По методу математической индукции заключаем, что утверждение
        справедливо для любого дерева.
    \end{enumerate}
\end{proof}

Задача №1. В любом дереве, в котором есть хотя бы одно ребро, есть висячие
вершины.

Задача №2. Висячая вершина не является шарниром.