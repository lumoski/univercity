\section{Хроматический многочлен, его свойства и методы нахождения. Найти хроматический многочлен 
графа.}

\begin{definition}
    Пусть $G=(V,E)$ -- некоторый граф и $t$ -- заданное число цветов.
    Число способов правильной раскраски графа $G$ с возможностью
    использования $t$ цветов называется его \textit{хроматическим многочленом} и
    обозначается $P(G,t)$.
\end{definition}

Рассмотрим хроматические многочлены для некоторых графов.
\begin{enumerate}[left=0.0em, labelsep=1em, topsep=0.0em, itemsep=0pt, parsep=0.5em]
    \item $P(G,t)$ графа из одной вершины без ребер равен $t$.
    \item $P(G,t)$ графа из двух вершин без ребер равен $t^2$.
    \item $P(G,t)$ графа из $n$ вершин без ребер (нульграфа) равен $t^n$.
    \item $P(G,t)$ полного $n$-вершинного графа $K_n$ равен
    
    $t(t-1)(t-2) \dots (t-(n-1))$.
    \item $P(G,t)$ дерева с $n$ вершинами равен $t(t-1)^{n-1}$.
    \item $P(G,t)$ графа из $n$ вершин, который является цепью равен $t(t-1)^{n-1}$.
    \item $P(G,t)$ графа из $n$ вершин, который является циклом, равен
    
    $(t-1)^n +(-1)^n(t-1)$.
\end{enumerate}


\textbf{Теоремы Зыкова}

Две теоремы, рассмотренные ниже, позволяют находить хроматический
многочлен графа.

Рассмотрим три операции на графах: удаление ребра, добавление ребра и
стягивание двух несмежных вершин.

В первой операции у графа из множества ребер удаляется выбранное ребро
$(u, v)$, граф в этом случае обозначается $G \setminus \set{(u, v)}$.

Во второй операции добавляется ребро $(u, v)$, которого не было, граф в этом
случае обозначается $G \cup \set{(u, v)}$.

В третьей операции две вершины $u, v$ отождествляются, вместо них берется
новая вершина, эта вершина в новом графе смежна всем вершинам,
которым смежна хотя бы одна из вершин $u, v$, граф в этом случае
обозначается $G/(uv)$.

Хроматический многочлен можно находить с помощью двух теорем.

\begin{theorem}
    \textbf{(Зыкова 1)}. Пусть $(u, v)$ -- ребро графа $G$, $t$ -- количество цветов,
    используемых при раскраске. Тогда
    \begin{align*}
        P(G,t) + P(G/(uv), t) = P(G \setminus \set{(u,v)},t).
    \end{align*}
\end{theorem}

\begin{proof}
    Выражение $P(G \setminus \set{(u,v)},t)$ в правой части равенства -- это
    число возможных правильных раскрасок графа $G$ c удаленным ребром
    $(u,v)$, поэтому в нем вершины u и v могут иметь как одинаковый, так и
    разные цвета.
    
    В левой части равенства слагаемое $P(G,t)$ -- это число различных
    правильных раскрасок графа $G$, в которых смежные вершины u и v имеют
    разные цвета, в слагаемом $P(G/(uv), t)$ вершины $u$ и $v$ стянуты в одну,
    поэтому имеют один цвет. Поэтому равенство справедливо.
    
    Если равенство записать в виде $P(G,t) = P(G \setminus \set{(u,v)},t) - P(G/(uv), t)$,
    
    то его можно использовать для нахождения хроматического многочлена,
    каждый раз производя операции удаления ребра и стягивания
    
    Этот алгоритм называют \textit{редукцией по нуль-графам} или \textit{алгоритмом
    Зыкова}.
\end{proof}

\begin{theorem}
    \textbf{(Зыкова 2)}. Пусть $u, v$ -- несмежные вершины графа $G$. Тогда
    \begin{align*}
        P(G,t) = P(G \cup \set{(u,v)},t) + P(G/(uv), t)
    \end{align*}
\end{theorem}

Применение этой теоремы для нахождения хроматического многочлена
называется редукцией по полным графам.

%TODO: ПРИМЕР

\newpage
Пусть хроматический многочлен графа $G$ имеет вид:
\begin{align*}
    P(G,t) = a_nt^n + a_{n-1}t^{n-1}+ \dots +a_1t+a_0.
\end{align*}

Справедливы следующие свойства хроматического многочлена:
\begin{enumerate}[left=0.0em, labelsep=1em, topsep=0.0em, itemsep=0pt, parsep=0.5em]
    \item Свободный член хроматического многочлена $a_0$ равен нулю.
    \item Старший член хроматического многочлена $a_n$ равен 1.
    \item Степень хроматического многочлена равна числу вершин в графе.
    \item Пусть $v$ -- висячая вершина графа $G$, пусть $G_1$ -- граф, полученный из
    графа $G$ удалением этой вершины и инцидентного ей ребра. Тогда
    $P(G,t) = (t-1) P(G_1,t)$
    \item Пусть $v$ -- вершина графа $G$ степени два, смежная с двумя вершинами $u$
    и $w$, которые, в свою очередь, смежны друг с другом (то есть три эти
    вершины образуют граф $K_3$). Пусть $G_1$ -- граф, полученный из графа $G$
    удалением этой вершины и двух инцидентных ей ребер. Тогда
    $P(G,t) = (t-2) P(G_1,t)$
    \item Пусть $G_1, G_2, \dots, G_s$ -- компоненты связности графа $G$.
    
    Тогда $P(G,t) = \prod_{i=1}^{s}(G_i, t)$
\end{enumerate}