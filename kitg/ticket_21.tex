\section{Алгоритм Флери нахождения эйлерова цикла. Применить к указанному графу.}

Рассмотрим один из алгоритмов нахождения эйлерова цикла
-- \textit{алгоритм Флёри}. Он отличается от приведенного алгоритма в
доказательстве теоремы Эйлера тем, что каждый раз при добавлении нового
ребра в марщрут нужно проверять, не является ли это ребро мостом.

В алгоритме нумеруются все ребра графа числами $1,2, \dots , |E|$, так, чтобы
номер, присвоенный данному ребру, указывал, каким по счету это ребро будет
в эйлеровом цикле.

\begin{enumerate}[left=0.0em, labelsep=1em, topsep=0.0em, itemsep=0pt, parsep=0.5em]
    \item Выбираем произвольную вершину $u$, присваиваем произвольному
    ребру $(u,v)$ номер $k=1$. Вычеркиваем это ребро из множества ребер графа и
    переходим в вершину $v$.
    \item Пусть $v$ -- вершина, в которую мы перешли в результате выполнения
    предыдущего шага, $k$ -- номер, присвоенный ребру на этом шаге. Выбираем
    любое ребро, инцидентное вершине v. При этом мост выбираем только в
    случае, когда ребер, инцидентных вершине и не являющихся мостами нет.
    Этому ребру присваиваем номер $k+1$, проходим по нему в следующую
    вершину и вычеркиваем это ребро.
    \item Если в графе есть не вычеркнутые ребра, то переходим к шагу 2, иначе
    stop.
\end{enumerate}

\begin{definition}
    Граф называется \textit{полуэйлеровым}, если в нём существует цель,
    проходящая по всем рёбрам графа.
\end{definition}

\begin{theorem}
    Граф без изолированных вершин называется \textit{полуэйлеровым графом}
    тогда и только тогда, когда выполняется два условия:
    \begin{enumerate}[left=0.0em, labelsep=1em, topsep=0.0em, itemsep=0pt, parsep=0.5em]
        \item Граф $G$ является связным
        \item Только две вершины в графе имеют нечетную степень.
    \end{enumerate}
\end{theorem}

\begin{definition}
    \textit{Полустепенью исхода вершины $v$} орграфа $\delta^-(v)$ называется количество
    рёбер, исходящих из данной вершины.
\end{definition}

\begin{definition}
    \textit{Полустепенью захода вершины $v$}
    орграфа $\delta^+(v)$ называют количество рёбер, заходящих в данную вершину.
\end{definition}

\newpage
Критерий эйлерова и полуэйлерова графа можно дать и для
ориентированного графа без изолированных вершин: для случая
эйлерова графа: полустепени исхода всех вершин должны равняться ее
полустепени захода, граф должен быть сильносвязным; для случая
полуэйлерова графа: полустепени исхода всех вершин, кроме двух
должны равняться полустепени захода, для одной из оставшихся вершин
полустепень исхода на единицу больше полустепени захода, для другой
вершины наоборот.