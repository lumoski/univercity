\section{Теорема об укладке графа в трехмерном пространстве.}

\begin{theorem}
    Любой конечный граф можно уложить в трёхмерном пространстве.
\end{theorem}

\begin{proof}
    Рассмотрим произвольный граф с $n$ вершинами и $r$ рёбрами.
    Проведем прямую, отметим на ней $n$ различных точек. Каждой из них
    сопоставим вершину графа. Через прямую проведем $r$ различных плоскостей,
    каждой плоскости сопоставим ребро графа. Каждому ребру $(u,v)$ сопоставим
    полуокружность, соединяющую вершины $u$ и $v$, и проходящую в плоскости,
    соответствующей данному ребру. Очевидно, что такие ребра не будут
    пересекаться во внутренних точках.
\end{proof}