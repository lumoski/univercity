\section{Разбиение множества. Доказать теорему о связи фактор-множества множества по отношению 
эквивалентности и разбиением множества.}

\begin{definition}
    \textit{Разбиением множества $A$} называется такое семейство его
    непустых подмножеств, что их объединение совпадает с множеством $A$, а
    пересечение двух различных подмножеств является пустым множеством.
\end{definition}

Пример: \\
$A = \set{1,2,3,4,5}$, Множество подмножеств множества $A$ \\
$\set{\set{1,2,3},\set{2,3},\set{4,5}}$ не является разбиением множества $A$, так как элементы
2,3 принадлежат одновременно двум подмножествам.
Множество подмножеств множества $A$ $\set{\set{1},\set{2,3},\set{4,5}}$ является разбиением
множества $A$ по определению.

\begin{theorem}
    Пусть $R$ - отношение эквивалентности на множестве $A$. Фактор
    множество множества $A$ по $R$ задает разбиение этого множества.
    (подмножествами разбиения являются классы эквивалентности).
    \begin{align*}
        A/R=\set{\overline{a} | a \in A}
    \end{align*}
\end{theorem}

\begin{proof}
    Докажем, что классы эквивалентности по отношению $R$ являются
    подмножествами разбиения множества.
    \begin{enumerate}[left=0.0em, labelsep=1em, topsep=0.0em, itemsep=0pt, parsep=0.5em]
        \item Докажем, что каждый класс эквивалентности не является пустым:
        \begin{align*}
            \overline{a} &= \set{x \in A | xRa} \\
            R &- \text{рефлексивно} \Rightarrow aRa \Rightarrow a \in \overline{a}
        \end{align*}
        \item Докажем: если элемент $c \in \overline{a}$ и $c \in \overline{b}$,
        то эти классы совпадают: $\overline{a}=\overline{b}$.\\
        Это означает, что различные классы не пересекаются
        \\ $c \in \overline{a} \Rightarrow cRa = aRc$ \\
        $c \in \overline{b} \Rightarrow cRb = bRc$ \\
        $aRb \Rightarrow bRa$ \\
        Это означает, что если какой-то элемент принадлежит двум классам
        эквивалентности, то элементы, порождающие эти классы, находятся в этом
        отношении.
    \end{enumerate}
    \newpage
    Докажем теперь, что $\overline{a}=\overline{b}$. Для этого докажем два включения.
    \begin{enumerate}[left=0.0em, labelsep=1em, topsep=0.0em, itemsep=0pt, parsep=0.5em]
        \item $\overline{a} \subset \overline{b}$ \\
        Докажем: $(\forall x)(x \in \overline{a} \Rightarrow x \in \overline{b})$
        \begin{align*}
            x \in \overline{a} \Rightarrow xRa \wedge aRb \Rightarrow xRb \Rightarrow x \in \overline{b}
        \end{align*}
        \item $\overline{b} \subset \overline{a}$ \\
        Докажем: $(\forall x)(x \in \overline{b} \Rightarrow x \in \overline{a})$
        \begin{align*}
            x \in \overline{b} \Rightarrow xRb \wedge bRa \Rightarrow xRa \Rightarrow x \in \overline{a}
        \end{align*}
    \end{enumerate}
    Докажем, что $\cup \overline{a} = A$. Так как $(\forall a \in A) a \in \overline{a}$,
    то $A \subset \cup \overline{a}$, \\ любой класс эквивалентности по определению является подмножеством
    множества $A$, поэтому и объединение классов является подмножеством $A$.
    По определению получили, что фактор множество является разбиением
    множества $a$.
\end{proof}

Из этой теоремы следует следующее:
\begin{enumerate}[left=0.0em, labelsep=1em, topsep=0.0em, itemsep=0pt, parsep=0.5em]
    \item $(\forall a \in A) a \in \overline{a}$
    \item $(\forall a,b \in A) (a \in \overline{b} \Leftrightarrow \overline{a} = \overline{b})$
    \item $(\forall a,b \in A) (a \notin \overline{b} \Leftrightarrow \overline{a} \cap \overline{b} = \varnothing)$
    \item $(\forall a \in A) \cup \overline{a} = A$.
\end{enumerate}